\documentclass{article}
\usepackage{amsmath}
\usepackage{amssymb}
\usepackage{enumitem}
\usepackage{tikz}
\usetikzlibrary{graphs}
\usepackage[utf8]{inputenc}
\usepackage[T5]{fontenc}
\usepackage[vietnamese]{babel}

\title{Combinatorics And Graph Theory-Final}
\author{Phạm Phước Minh Hiếu}

\begin{document}
	\section{Project 1: Mathematical Induction {\it\&} Recurrence Relations -- Đồ Án 1: Quy Nạp Toán Học {\it\&} Quan Hệ Truy Hồi}
	
	\subsection{Mathematical Induction -- Quy nạp toán học}
	
	\textbf{Định lý.} Với mọi $n \in \mathbb{N},\ n \ge 1$, ta có:
	\[
	\sum_{k=1}^{n} k = \frac{n(n+1)}{2}.
	\]
	
	\textbf{Chứng minh:} Sử dụng phương pháp quy nạp toán học.
	
	\begin{itemize}[leftmargin=1.5cm]
		\item \textbf{Cơ sở quy nạp:} Với $n = 1$:
		\[
		\sum_{k=1}^{1} k = 1 = \frac{1(1+1)}{2} = 1.
		\]
		Đúng.
		
		\item \textbf{Giả thiết quy nạp:} Giả sử công thức đúng với $n = k$, tức là:
		\[
		\sum_{i=1}^{k} i = \frac{k(k+1)}{2}.
		\]
		
		\item \textbf{Bước quy nạp:} Xét $n = k + 1$:
		\begin{align*}
			\sum_{i=1}^{k+1} i &= \left(\sum_{i=1}^{k} i\right) + (k+1) \\
			&= \frac{k(k+1)}{2} + (k+1) \quad \text{(theo giả thiết quy nạp)}\\
			&= \frac{k(k+1) + 2(k+1)}{2} \\
			&= \frac{(k+1)(k + 2)}{2}.
		\end{align*}
		Đây chính là công thức với $n = k + 1$.
		
		\item \textbf{Kết luận:} Theo nguyên lý quy nạp, mệnh đề được chứng minh đúng với mọi $n \in \mathbb{N},\ n \ge 1$.
	\end{itemize}
	
	\subsection{Recurrence Relation -- Quan hệ truy hồi}
	
	\textbf{Đề bài:} Xét dãy truy hồi:
	\[
	a_1 = 1,\quad a_n = a_{n-1} + 2n - 1 \quad \text{với } n \ge 2.
	\]
	
	\textbf{Tìm công thức tổng quát của $a_n$ theo $n$.}
	
	\textbf{Giải:}
	
	Ta tính một vài giá trị đầu:
	\[
	a_1 = 1,\quad a_2 = 1 + 3 = 4,\quad a_3 = 4 + 5 = 9,\quad a_4 = 9 + 7 = 16.
	\]
	
	Dễ thấy:
	\[
	a_n = n^2.
	\]
	
	\textbf{Chứng minh bằng quy nạp:}
	
	\begin{itemize}[leftmargin=1.5cm]
		\item \textbf{Cơ sở:} $n = 1$: $a_1 = 1 = 1^2$, đúng.
		
		\item \textbf{Giả thiết quy nạp:} Giả sử $a_k = k^2$ đúng với một $k \ge 1$.
		
		\item \textbf{Bước quy nạp:}
		\[
		a_{k+1} = a_k + 2(k+1) - 1 = k^2 + 2k + 1 = (k+1)^2.
		\]
		
		\item \textbf{Kết luận:} $a_n = n^2$ với mọi $n \in \mathbb{N},\ n \ge 1$.
	\end{itemize}
	
	\section{Project 2: Counting, Probability, Balls, \& Boxes -- Đồ Án 2: Đếm, Xác Suất, Banh \& Hộp}
	
	\textbf{Cho 5 quả banh phân biệt và 3 hộp phân biệt.}
	
	\begin{itemize}[leftmargin=1.5cm]
		\item[(a)] Có bao nhiêu cách đặt 5 quả banh vào 3 hộp?
		\item[(b)] Nếu mỗi quả banh được đặt ngẫu nhiên vào một hộp (các quả độc lập, xác suất như nhau), xác suất có ít nhất một hộp trống là bao nhiêu?
	\end{itemize}
	
	\subsection*{Câu (a):}
	Mỗi quả banh có 3 lựa chọn để vào một trong 3 hộp.
	
	Vì các quả banh là phân biệt và được đặt độc lập:
	\[
	\text{Tổng số cách} = 3^5 = 243.
	\]
	
	\subsection*{Câu (b):}
	Ta cần tính:
	\[
	\mathbb{P}(\text{ít nhất một hộp trống}) = 1 - \mathbb{P}(\text{không có hộp nào trống}).
	\]
	
	\textbf{Gọi $A$ là biến cố “không có hộp nào trống”}
	
	Áp dụng nguyên lý bao hàm – loại trừ (Inclusion–Exclusion), ta tính số cách phân chia 5 quả banh (phân biệt) vào 3 hộp (phân biệt) sao cho mỗi hộp có ít nhất 1 quả (không rỗng).
	
	Số cách:
	\[
	N = 3^5 - \binom{3}{1} 2^5 + \binom{3}{2} 1^5 = 243 - 3 \cdot 32 + 3 \cdot 1 = 243 - 96 + 3 = 150.
	\]
	
	Vậy xác suất để không hộp nào trống là:
	\[
	\mathbb{P}(A) = \frac{150}{243}.
	\]
	
	Vậy xác suất có ít nhất một hộp trống:
	\[
	1 - \frac{150}{243} = \frac{93}{243} = \frac{31}{81}.
	\]
	
\end{document}