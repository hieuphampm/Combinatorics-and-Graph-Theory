\documentclass{article}
\usepackage{amsmath}
\usepackage{amssymb}
\usepackage{enumitem}
\usepackage{fancyvrb}
\usepackage{booktabs}
\usepackage{tikz}
\usetikzlibrary{graphs}
\usepackage[utf8]{inputenc}
\usepackage[T5]{fontenc}
\usepackage[vietnamese]{babel}

\title{Combinatorics And Graph Theory-Final}
\author{Phạm Phước Minh Hiếu}

\begin{document}
	\section*{Project 1: Mathematical Induction {\it\&} Recurrence Relations -- Đồ Án 1: Quy Nạp Toán Học {\it\&} Quan Hệ Truy Hồi}
	
	\subsection{Mathematical Induction -- Quy nạp toán học}
	
	\textbf{Định lý.} Với mọi $n \in \mathbb{N},\ n \ge 1$, ta có:
	\[
	\sum_{k=1}^{n} k = \frac{n(n+1)}{2}.
	\]
	
	\textbf{Chứng minh:} Sử dụng phương pháp quy nạp toán học.
	
	\begin{itemize}[leftmargin=1.5cm]
		\item \textbf{Cơ sở quy nạp:} Với $n = 1$:
		\[
		\sum_{k=1}^{1} k = 1 = \frac{1(1+1)}{2} = 1.
		\]
		Đúng.
		
		\item \textbf{Giả thiết quy nạp:} Giả sử công thức đúng với $n = k$, tức là:
		\[
		\sum_{i=1}^{k} i = \frac{k(k+1)}{2}.
		\]
		
		\item \textbf{Bước quy nạp:} Xét $n = k + 1$:
		\begin{align*}
			\sum_{i=1}^{k+1} i &= \left(\sum_{i=1}^{k} i\right) + (k+1) \\
			&= \frac{k(k+1)}{2} + (k+1) \quad \text{(theo giả thiết quy nạp)}\\
			&= \frac{k(k+1) + 2(k+1)}{2} \\
			&= \frac{(k+1)(k + 2)}{2}.
		\end{align*}
		Đây chính là công thức với $n = k + 1$.
		
		\item \textbf{Kết luận:} Theo nguyên lý quy nạp, mệnh đề được chứng minh đúng với mọi $n \in \mathbb{N},\ n \ge 1$.
	\end{itemize}
	
	\subsection{Recurrence Relation -- Quan hệ truy hồi}
	
	\textbf{Đề bài:} Xét dãy truy hồi:
	\[
	a_1 = 1,\quad a_n = a_{n-1} + 2n - 1 \quad \text{với } n \ge 2.
	\]
	
	\textbf{Tìm công thức tổng quát của $a_n$ theo $n$.}
	
	\textbf{Giải:}
	
	Ta tính một vài giá trị đầu:
	\[
	a_1 = 1,\quad a_2 = 1 + 3 = 4,\quad a_3 = 4 + 5 = 9,\quad a_4 = 9 + 7 = 16.
	\]
	
	Dễ thấy:
	\[
	a_n = n^2.
	\]
	
	\textbf{Chứng minh bằng quy nạp:}
	
	\begin{itemize}[leftmargin=1.5cm]
		\item \textbf{Cơ sở:} $n = 1$: $a_1 = 1 = 1^2$, đúng.
		
		\item \textbf{Giả thiết quy nạp:} Giả sử $a_k = k^2$ đúng với một $k \ge 1$.
		
		\item \textbf{Bước quy nạp:}
		\[
		a_{k+1} = a_k + 2(k+1) - 1 = k^2 + 2k + 1 = (k+1)^2.
		\]
		
		\item \textbf{Kết luận:} $a_n = n^2$ với mọi $n \in \mathbb{N},\ n \ge 1$.
	\end{itemize}
	
	\section*{Project 2: Counting, Probability, Balls, \& Boxes -- Đồ Án 2: Đếm, Xác Suất, Banh \& Hộp}
	
	\textbf{Cho 5 quả banh phân biệt và 3 hộp phân biệt.}
	
	\begin{itemize}[leftmargin=1.5cm]
		\item[(a)] Có bao nhiêu cách đặt 5 quả banh vào 3 hộp?
		\item[(b)] Nếu mỗi quả banh được đặt ngẫu nhiên vào một hộp (các quả độc lập, xác suất như nhau), xác suất có ít nhất một hộp trống là bao nhiêu?
	\end{itemize}
	
	\subsection*{Câu (a):}
	Mỗi quả banh có 3 lựa chọn để vào một trong 3 hộp.
	
	Vì các quả banh là phân biệt và được đặt độc lập:
	\[
	\text{Tổng số cách} = 3^5 = 243.
	\]
	
	\subsection*{Câu (b):}
	Ta cần tính:
	\[
	\mathbb{P}(\text{ít nhất một hộp trống}) = 1 - \mathbb{P}(\text{không có hộp nào trống}).
	\]
	
	\textbf{Gọi $A$ là biến cố “không có hộp nào trống”}
	
	Áp dụng nguyên lý bao hàm – loại trừ (Inclusion–Exclusion), ta tính số cách phân chia 5 quả banh (phân biệt) vào 3 hộp (phân biệt) sao cho mỗi hộp có ít nhất 1 quả (không rỗng).
	
	Số cách:
	\[
	N = 3^5 - \binom{3}{1} 2^5 + \binom{3}{2} 1^5 = 243 - 3 \cdot 32 + 3 \cdot 1 = 243 - 96 + 3 = 150.
	\]
	
	Vậy xác suất để không hộp nào trống là:
	\[
	\mathbb{P}(A) = \frac{150}{243}.
	\]
	
	Vậy xác suất có ít nhất một hộp trống:
	\[
	1 - \frac{150}{243} = \frac{93}{243} = \frac{31}{81} \approx 0.38
	\]
	
	\section*{Project 3: Generating Functions -- Đồ Án 3: Hàm Sinh}
	
	\textbf{Cho 1 ví dụ:}
		Cho vô hạn số đồng xu có mệnh giá: 1, 2 và 5.  
	Hỏi có bao nhiêu cách chọn các đồng xu (không giới hạn số lượng mỗi loại) để tổng tiền bằng 10?
	
	\subsection*{Hàm sinh của từng loại đồng xu:}
	
	\begin{itemize}
		\item Với đồng xu 1: $1 + x + x^2 + x^3 + \cdots = \dfrac{1}{1 - x}$  
		\item Với đồng xu 2: $1 + x^2 + x^4 + x^6 + \cdots = \dfrac{1}{1 - x^2}$
		\item Với đồng xu 5: $1 + x^5 + x^{10} + x^{15} + \cdots = \dfrac{1}{1 - x^5}$
	\end{itemize}
	
	\subsection*{Hàm sinh tổng hợp:}
	
	Hàm sinh đếm số cách chọn các đồng xu để tổng là $n$:
	\[
	G(x) = \frac{1}{(1 - x)(1 - x^2)(1 - x^5)}
	\]
	
	Ta cần tìm hệ số của $x^{10}$ trong khai triển của $G(x)$:
	\[
	[x^{10}]\,G(x) = ?
	\]
	
	Với $[x^{10}]$:
	
	Ta sẽ đếm số cách chọn bộ số \((a, b, c)\) nguyên không âm sao cho:
	\[
	a + 2b + 5c = 10.
	\]
	
	\noindent\textbf{Trường hợp \(c = 0\):} \(a + 2b = 10\)
	
	\begin{itemize}[leftmargin=1.5cm]
		\item \(b = 0 \Rightarrow a = 10\)
		\item \(b = 1 \Rightarrow a = 8\)
		\item \(b = 2 \Rightarrow a = 6\)
		\item \(b = 3 \Rightarrow a = 4\)
		\item \(b = 4 \Rightarrow a = 2\)
		\item \(b = 5 \Rightarrow a = 0\)
	\end{itemize}
	Có \textbf{6 cách}.
	
	\noindent\textbf{Trường hợp \(c = 1\):} \(a + 2b = 5\)
	
	\begin{itemize}[leftmargin=1.5cm]
		\item \(b = 0 \Rightarrow a = 5\)
		\item \(b = 1 \Rightarrow a = 3\)
		\item \(b = 2 \Rightarrow a = 1\)
	\end{itemize}
	Có \textbf{3 cách}.
	
	\noindent\textbf{Trường hợp \(c = 2\):} \(a + 2b = 0\) \\
	Chỉ có 1 nghiệm: \(a = 0, b = 0\)
	
	\subsection*{3.2. Tổng số cách}
	
	Tổng cộng có:
	\[
	6 + 3 + 1 = \boxed{10} \text{ cách}.
	\]
	
	$\Rightarrow [x^{10}]\,G(x) = 10.$
	
	\section*{Project: Integer Partition -- Đồ Án: Phân Hoạch Số Nguyên}
	
	\subsection*{\underline{Bài toán 1:} Ferrers \& Ferrers transpose diagrams -- Biểu đồ Ferrers \& biểu đồ Ferrers chuyển vị}
	
	Nhập $n,k\in\mathbb{N}$. Viết chương trình {\sf C{\tt/}C++, Python} để in ra $p_k(n)$ biểu đồ Ferrers $F$ \& biểu đồ Ferrers chuyển vị $F^\top$ cho mỗi phân hoạch $\boldsymbol{\lambda} = (\lambda_1,\lambda_2,\ldots,\lambda_k)\in(\mathbb{N}^\star)^k$ có định dạng các dấu chấm được biểu diễn bởi dấu {\tt*}.
	
	\textbf{Ví dụ:}
	
	Cho $n = 5$, $k = 2$. Các phân hoạch của 5 thành đúng 2 phần tử là:
	
	\begin{itemize}
		\item $(4,1)$
		\item $(3,2)$
	\end{itemize}
	
	Biểu diễn Ferrers và Ferrers chuyển vị:
	
	\subsubsection*{Phân hoạch $(4,1)$}
	
	\textbf{Ferrers:}
	\begin{Verbatim}
		****
		*
	\end{Verbatim}
	
	\textbf{Ferrers chuyển vị:}
	\begin{Verbatim}
		**
		*
		*
		*
	\end{Verbatim}
	
	\subsubsection*{Phân hoạch $(3,2)$}
	
	\textbf{Ferrers:}
	\begin{Verbatim}
		***
		**
	\end{Verbatim}
	
	\textbf{Ferrers chuyển vị:}
	\begin{Verbatim}
		**
		**
		*
	\end{Verbatim}
	
	\subsection*{\underline{Bài toán 2:} Nhập $n,k\in\mathbb{N}$. Đếm số phân hoạch của $n\in\mathbb{N}$. Viết chương trình {\sf C{\tt/}C++, Python} để đếm số phân hoạch $p_{\max}(n,k)$ của $n$ sao cho phần tử lớn nhất là $k$. So sánh $p_k(n)$ \& $p_{\max}(n,k)$.
	}
	
	\textbf{Theo đề bài:} Cho $n, k \in \mathbb{N}$. Đếm:
	\begin{itemize}
		\item $p_k(n)$: số phân hoạch của $n$ thành đúng $k$ số nguyên dương.
		\item $p_{\max}(n, k)$: số phân hoạch của $n$ sao cho phần tử lớn nhất đúng bằng $k$.
	\end{itemize}
	
	\begin{itemize}
		\item $p_k(n)$: Số phân hoạch của $n$ thành đúng $k$ số nguyên dương, tức là:
		\[
		n = \lambda_1 + \lambda_2 + \cdots + \lambda_k,\quad \lambda_i \in \mathbb{N}^+
		\]
		\item $p_{\max}(n, k)$: Số phân hoạch của $n$ sao cho phần tử lớn nhất trong phân hoạch bằng đúng $k$.
	\end{itemize}
	
	\textbf{Cho một ví dụ với $n=5$}
	
	\begin{itemize}
		\item Các phân hoạch thành đúng $k=2$ phần tử:
		\[
		(4,1), (3,2)
		\quad\Rightarrow\quad p_2(5) = 2
		\]
		\item Các phân hoạch của $n=5$ có phần tử lớn nhất là $k=2$:
		\[
		(2,2,1), (2,1,1,1), (1,1,1,1,1)
		\quad\Rightarrow\quad p_{\max}(5,2) = 3
		\]
	\end{itemize}
	
	\[
	p_2(5) = 2 \quad\text{vs.}\quad p_{\max}(5,2) = 3
	\]
	\newpage
	\textbf{Thử so sánh với $p_2$(n) và $p_{max}$(n,2)}
	\begin{center}
		\begin{tabular}{@{}ccc@{}}
			\toprule
			$n$ & $p_2(n)$ & $p_{\max}(n,2)$ \\
			\midrule
			1 & 0 & 0 \\
			2 & 1 & 1 \\
			3 & 1 & 1 \\
			4 & 2 & 2 \\
			5 & 2 & 3 \\
			6 & 3 & 4 \\
			\bottomrule
		\end{tabular}
	\end{center}
\end{document}