\documentclass{article}
\usepackage{amsmath}
\usepackage{amssymb}
\usepackage{enumitem}
\usepackage{fancyvrb}
\usepackage{booktabs}
\usepackage{tikz}
\usepackage{graphicx}
\usepackage{algorithm}
\usepackage{algpseudocode}
\usetikzlibrary{graphs}
\usepackage[utf8]{inputenc}
\usepackage[T5]{fontenc}
\usepackage[vietnamese]{babel}

\title{Combinatorics And Graph Theory-Final}
\author{Phạm Phước Minh Hiếu - 2201700085}

\begin{document}
	\maketitle
	\section*{Project: Integer Partition -- Đồ Án: Phân Hoạch Số Nguyên}
	
	\subsection*{\underline{Bài toán 1:} Ferrers \& Ferrers transpose diagrams -- Biểu đồ Ferrers \& biểu đồ Ferrers chuyển vị}
	
	Nhập $n,k\in\mathbb{N}$. Viết chương trình {\sf C{\tt/}C++, Python} để in ra $p_k(n)$ biểu đồ Ferrers $F$ \& biểu đồ Ferrers chuyển vị $F^\top$ cho mỗi phân hoạch $\boldsymbol{\lambda} = (\lambda_1,\lambda_2,\ldots,\lambda_k)\in(\mathbb{N}^\star)^k$ có định dạng các dấu chấm được biểu diễn bởi dấu {\tt*}.
	
	\textbf{Ví dụ:}
	
	Cho $n = 5$, $k = 2$. Các phân hoạch của 5 thành đúng 2 phần tử là:
	
	\begin{itemize}
		\item $(4,1)$
		\item $(3,2)$
	\end{itemize}
	
	Biểu diễn Ferrers và Ferrers chuyển vị:
	
	\subsubsection*{Phân hoạch $(4,1)$}
	
	\textbf{Ferrers:}
	\begin{Verbatim}
		****
		*
	\end{Verbatim}
	
	\textbf{Ferrers chuyển vị:}
	\begin{Verbatim}
		**
		*
		*
		*
	\end{Verbatim}
	
	\subsubsection*{Phân hoạch $(3,2)$}
	
	\textbf{Ferrers:}
	\begin{Verbatim}
		***
		**
	\end{Verbatim}
	
	\textbf{Ferrers chuyển vị:}
	\begin{Verbatim}
		**
		**
		*
	\end{Verbatim}
	
	\subsection*{\underline{Bài toán 2:} Nhập $n,k\in\mathbb{N}$. Đếm số phân hoạch của $n\in\mathbb{N}$. Viết chương trình {\sf C{\tt/}C++, Python} để đếm số phân hoạch $p_{\max}(n,k)$ của $n$ sao cho phần tử lớn nhất là $k$. So sánh $p_k(n)$ \& $p_{\max}(n,k)$.
	}
	
	\textbf{Theo đề bài:} Cho $n, k \in \mathbb{N}$. Đếm:
	\begin{itemize}
		\item $p_k(n)$: số phân hoạch của $n$ thành đúng $k$ số nguyên dương.
		\item $p_{\max}(n, k)$: số phân hoạch của $n$ sao cho phần tử lớn nhất đúng bằng $k$.
	\end{itemize}
	
	\textbf{Cho một ví dụ với $n=5$}
	
	\begin{itemize}
		\item Các phân hoạch thành đúng $k=2$ phần tử:
		\[
		(4,1), (3,2)
		\quad\Rightarrow\quad p_2(5) = 2
		\]
		\item Các phân hoạch của $n=5$ có phần tử lớn nhất là $k=2$:
		\[
		(2,2,1), (1,1,1,2)
		\quad\Rightarrow\quad p_{\max}(5,2) = 2
		\]
	\end{itemize}
	
	\[
	p_2(5) = 2 \quad\text{vs.}\quad p_{\max}(5,2) = 2
	\]
	\newpage
	\textbf{Bảng so sánh \( p_2(n) \) và \( p_{\max}(n, 2) \):}
	
	\[
	\begin{array}{|c|c|c|}
		\hline
		n & p_2(n) & p_{\max}(n, 2) \\
		\hline
		1 & 0 & 0 \\
		2 & 1 & 1 \\
		3 & 1 & 1 \\
		4 & 2 & 2 \\
		5 & 2 & 1 \\
		6 & 3 & 2 \\
		7 & 3 & 3 \\
		8 & 4 & 4 \\
		9 & 4 & 4 \\
		10 & 5 & 5 \\
		\hline
	\end{array}
	\]
	
	\subsection*{Số phân hoạch tự liên hợp}
	 Nhập $n,k\in\mathbb{N}$. 
	 \begin{itemize}
	 	\item (a) Đếm số phân hoạch tự liên hợp của $n$ có $k$ phần, ký hiệu $p_k^{\rm selfcjg}(n)$, rồi in ra các phân hoạch đó.
	 	\item (b) Đếm số phân hoạch của $n$ có lẻ phần, rồi so sánh với $p_k^{\rm selfcjg}(n)$.
	 	\item (c) Thiết lập công thức truy hồi cho $p_k^{\rm selfcjg}(n)$, rồi implementation bằng: (i) đệ quy. (ii) quy hoạch động.
	 \end{itemize}
	   
	Cho $n,k\in\mathbb{N}$.
	
	\begin{itemize}
		\item[(a)] Đếm số phân hoạch tự liên hợp của $n$ có đúng $k$ phần, ký hiệu là $p_k^{\text{selfcjg}}(n)$ và liệt kê các phân hoạch đó.
		
		\textbf{Ví dụ:} Với $n=7$, các phân hoạch tự liên hợp gồm: 
		\[
		(4,3),\quad (3,3,1),\quad (2,2,2,1),\quad (1,1,1,1,1,1,1)
		\]
		Khi lọc theo số phần $k$, ta chỉ giữ các phân hoạch có đúng $k$ phần tử.
		
		\item[(b)] Đếm số phân hoạch của $n$ có số phần tử là số lẻ. So sánh giá trị đó với $p_k^{\text{selfcjg}}(n)$.  
		Ký hiệu tổng số phân hoạch có số phần lẻ là $q(n)$.
		
		\textbf{Định lý:} Tổng số phân hoạch tự liên hợp của $n$ đúng bằng số phân hoạch có số phần tử lẻ (Euler).
		
		\item[(c)] \textbf{Thiết lập công thức truy hồi cho $p_k^{\text{selfcjg}}(n)$}
		
		Gọi $p_k^{\text{selfcjg}}(n)$ là số phân hoạch \textbf{tự liên hợp} của $n$ thành đúng $k$ phần.
		
		Do một phân hoạch tự liên hợp luôn gồm các số lẻ không tăng và không vượt quá $k$, do đối xứng qua đường chéo chính của biểu đồ Ferrers.
		
		Ta có thể định nghĩa truy hồi như sau:
		
		\[
		p_k^{\text{selfcjg}}(n) =
		\begin{cases}
			1, & \text{nếu } n = 0 \text{ và } k = 0 \\
			0, & \text{nếu } n < 0 \text{ hoặc } k \leq 0 \\
			\sum\limits_{\substack{1 \leq m \leq \min(2k-1,n)\\ m \text{ lẻ}}} p_{k-1}^{\text{selfcjg}}(n - m), & \text{ngược lại}
		\end{cases}
		\]
		
		
		$\rightarrow$ Mỗi phần thêm vào là một số lẻ $m$, ta trừ nó khỏi $n$ và giảm số phần đi 1.
		
		$\rightarrow$ Ràng buộc số lẻ xuất hiện là do cấu trúc đối xứng: mỗi điểm ở trên đường chéo cần đối xứng với một điểm dưới đường chéo, nên số phần tử ở mỗi dòng phải lẻ.
		
		\textbf{Ngoài ra}, ta có thể thiết lập công thức không phụ thuộc vào $k$, khi đếm tổng số phân hoạch tự liên hợp (tức là $p^{\text{selfcjg}}(n) = \sum_k p_k^{\text{selfcjg}}(n)$):
		
		\[
		p^{\text{selfcjg}}(n) = \# \left\{ (\lambda_1, \lambda_2, \dots, \lambda_k) \mid \sum_{i=1}^k \lambda_i = n,\ \lambda_i \text{ lẻ},\ \lambda_1 \geq \lambda_2 \geq \cdots \geq \lambda_k,\ \lambda_i \geq i \right\}
		\]
		
		Trong đó điều kiện \( \lambda_i \geq i \) đảm bảo đối xứng qua đường chéo chính.
		
		\textbf{Công thức tạo hàm sinh:} Hàm sinh của số phân hoạch tự liên hợp là:
		
		\[
		\sum_{n=0}^\infty p^{\text{selfcjg}}(n) q^n = \prod_{k=1}^\infty (1 + q^{2k-1})
		\]
		
		Do mỗi phần tử trong phân hoạch tự liên hợp tương ứng với một chiều dài móc câu (hook length) lẻ.
		
		
	\end{itemize}
	
	\section*{Project 4: Graph \& Tree Traversing Problems -- Đồ Án 4: Các Bài Toán Duyệt Đồ Thị \& Cây}
	
	\subsection*{Viết chương trình {\sf C{\tt/}C++, Python} chuyển đổi giữa 4 dạng biểu diễn: adjacency matrix, adjacency list, extended adjacency list, adjacency map cho $3$ đồ thị: đơn đồ thị, đa đồ thị, đồ thị tổng quát; \& 3 dạng biểu diễn: array of parents, first-child next-sibling, graph-based representation of trees của cây.}
	
	\subsubsection*{1. Biểu diễn đồ thị}
	
	Một đồ thị có thể được biểu diễn theo nhiều cách, mỗi cách phục vụ một mục đích tính toán hoặc lưu trữ khác nhau. Các dạng biểu diễn phổ biến bao gồm:
	
	\begin{itemize}
		\item \textbf{Ma trận kề (Adjacency Matrix)}: Là một ma trận vuông kích thước $n \times n$, trong đó phần tử $A[i][j]$ chứa số cạnh hoặc trọng số từ đỉnh $i$ đến $j$.
		\item \textbf{Danh sách kề (Adjacency List)}: Là một danh sách trong đó mỗi phần tử tương ứng với một đỉnh, lưu các đỉnh kề và trọng số cạnh.
		\item \textbf{Danh sách kề mở rộng (Extended Adjacency List)}: Mở rộng từ danh sách kề, mỗi cạnh lưu thêm thông tin như số lượng cạnh (cho đa đồ thị), nhãn, hoặc trọng số.
		\item \textbf{Bản đồ kề (Adjacency Map)}: Mỗi đỉnh ánh xạ tới một \texttt{map} hoặc \texttt{dict} lưu các đỉnh kề cùng trọng số tương ứng. Cấu trúc này rất thuận tiện cho việc truy vấn cạnh hoặc trọng số nhanh.
	\end{itemize}

	Về ý tưởng chuyển đổi:
	
	\begin{itemize}
		\item \textbf{Từ Ma trận kề $\rightarrow$ Danh sách kề}:
		Duyệt toàn bộ ma trận, nếu $A[i][j] \neq 0$ thì thêm một cạnh từ $i$ đến $j$ vào danh sách.
		\item \textbf{Từ Danh sách kề $\rightarrow$ Ma trận kề}:
		Tạo một ma trận $n \times n$ toàn 0. Với mỗi đỉnh $i$ và danh sách cạnh $(j, w)$, gán $A[i][j] = w$.
		\item \textbf{Từ Danh sách kề $\rightarrow$ Bản đồ kề}:
		Với mỗi đỉnh $i$, ánh xạ mỗi đỉnh kề $j$ vào trọng số $w$ trong một bản đồ.
	\end{itemize}
	
	\subsubsection*{2. Biểu diễn cây}
	
	Một cây có thể được biểu diễn theo nhiều dạng:
	
	\begin{itemize}
		\item \textbf{Mảng cha (Array of Parents)}: Mỗi nút $i$ có một giá trị $parent[i]$, là chỉ số nút cha. Nếu $parent[i] = -1$ thì $i$ là gốc.
		\item \textbf{First-Child Next-Sibling (FCNS)}: Mỗi nút có 2 con trỏ: \texttt{firstChild} và \texttt{nextSibling}, cho phép biểu diễn cây dạng nhị phân hóa.
		\item \textbf{Đồ thị biểu diễn cây}: Xem cây là một đồ thị không chu trình có hướng từ cha đến con.
	\end{itemize}
	
	Về việc chuyển đổi từ mảng cha sang FCNS:
	
	\begin{enumerate}
		\item Khởi tạo danh sách nút FCNS.
		\item Duyệt từng nút $i$:
		\begin{itemize}
			\item Nếu $parent[i] = -1$ thì nút $i$ là gốc.
			\item Ngược lại:
			\begin{itemize}
				\item Nếu cha chưa có con, gán \texttt{firstChild = i}.
				\item Nếu đã có con, tìm con cuối cùng và gán \texttt{nextSibling = i}.
			\end{itemize}
		\end{itemize}
	\end{enumerate}

	\subsection*{Làm Problems 1.1--1.6 \& Exercises 1.1--1.10.}
	
	\subsubsection*{1.1  Determine the size of the complete graph Kn on n vertices and the complete bipartite graph Kp,q on p + q vertices.}
	
	\paragraph*{1.1.1. Đồ thị đầy đủ \( K_n \):}
	
	Đồ thị đầy đủ \( K_n \) là đồ thị đơn vô hướng trong đó mỗi cặp hai đỉnh phân biệt đều được nối với nhau bởi đúng một cạnh.
	
	Tổng số cạnh trong \( K_n \) chính là số cách chọn ra 2 đỉnh bất kỳ từ \( n \) đỉnh để nối thành một cạnh:
	\[
	\text{Số cạnh của } K_n = \binom{n}{2} = \frac{n(n-1)}{2}
	\]
	
	\paragraph*{1.1.2. Đồ thị hai phía đầy đủ \( K_{p,q} \):}
	
	Đồ thị hai phía đầy đủ \( K_{p,q} \) là đồ thị trong đó tập đỉnh được chia thành hai tập rời \( V_1 \) và \( V_2 \) với kích thước lần lượt là \( p \) và \( q \), và mỗi đỉnh trong \( V_1 \) được nối với mọi đỉnh trong \( V_2 \).
	
	Do đó, tổng số cạnh là:
	\[
	\text{Số cạnh của } K_{p,q} = p \cdot q
	\]
	
	\subsubsection*{1.2 Determine the values of n for which the circle graph Cn on n vertices is bipartite,  and also the values of n for which the complete graph Kn is bipartite.}
	
	\paragraph*{1.2.1. Đồ thị vòng \( C_n \):}
	
	Đồ thị vòng \( C_n \) là một đồ thị mà các đỉnh được nối thành một vòng khép kín. Một đồ thị là hai phía khi và chỉ khi nó không chứa chu trình lẻ.
	
	Như vậy: \( C_n \) chỉ chứa một chu trình có độ dài đúng bằng \( n \), nên:
	\begin{itemize}
		\item Nếu \( n \) chẵn thì \( C_n \) không chứa chu trình lẻ \( \Rightarrow \) là đồ thị hai phía.
		\item Nếu \( n \) lẻ thì \( C_n \) chứa chu trình lẻ \( \Rightarrow \) không là đồ thị hai phía.
	\end{itemize}
	
	$\rightarrow$ \( C_n \) là đồ thị hai phía khi và chỉ khi \( n \) là số chẵn:
	\[
	\boxed{C_n \text{ là đồ thị hai phía khi } n \equiv 0 \pmod{2}}
	\]
	
	\paragraph*{1.2.2. Đồ thị đầy đủ \( K_n \):}
	
	Đồ thị đầy đủ \( K_n \) là đồ thị mà mỗi cặp hai đỉnh phân biệt đều được nối bởi một cạnh.
	
	Như vậy: một đồ thị hai phía không thể chứa tam giác (chu trình độ dài 3), nhưng \( K_n \) với \( n \geq 3 \) luôn chứa tam giác (vì mọi bộ ba đỉnh bất kỳ đều tạo thành một tam giác).
	
	\begin{itemize}
		\item Với \( n = 1 \): \( K_1 \) không có cạnh nào, nên là đồ thị hai phía.
		\item Với \( n = 2 \): \( K_2 \) chỉ có 1 cạnh nối 2 đỉnh, rõ ràng là hai phía.
		\item Với \( n \geq 3 \): \( K_n \) chứa chu trình lẻ (tam giác), nên không phải hai phía.
	\end{itemize}
	
	Do đó: $K_n$ \text{ là đồ thị hai phía khi và chỉ khi } n $\leq$ 2
	
	\subsubsection*{1.3 Liệt kê tất cả các cây khung của đồ thị trong Hình 1.30 và đếm số lượng cây khung}
	
	\textbf{Bước 1: Chuyển sang đồ thị vô hướng}
	
	Ta bỏ hướng của tất cả các cung trong Hình 1.30 để thu được đồ thị vô hướng tương ứng. Tập các cạnh vô hướng là:
	
	\[
	\begin{aligned}
		E = \{ 
		& \{v_1, v_2\},\ \{v_1, v_4\},\ \{v_1, v_6\}, \ \{v_3, v_1\}, \\
		& \{v_2, v_4\},\ \{v_2, v_5\}, \\
		& \{v_3, v_6\}, \ \{v_4, v_6\},\ \{v_4, v_7\},\ \{v_6, v_7\} 
		\}
	\end{aligned}
	\]
	
	Số đỉnh: \( n = 7 \), số cạnh: \( m = 10 \).
	
	\vspace{1em}
	\textbf{Bước 2: Định nghĩa cây khung}
	
	Cây khung của một đồ thị vô hướng liên thông là một tập con của các cạnh sao cho:
	
	\begin{itemize}
		\item Tập các đỉnh của cây khung bao phủ toàn bộ đỉnh của đồ thị.
		\item Cây khung là một cây: liên thông và không có chu trình.
		\item Cây khung có đúng \( n - 1 = 6 \) cạnh (vì \( n = 7 \)).
	\end{itemize}
	
	\vspace{1em}
	\textbf{Bước 3: Sinh tất cả các tập con gồm 6 cạnh từ 10 cạnh}
	
	Tổng số tổ hợp các cạnh có thể chọn là:
	
	\[
	\binom{10}{6} = 210
	\]
	
	Với mỗi tập con 6 cạnh, ta kiểm tra sao cho đáp ứng:
	\begin{enumerate}
		\item Đồ thị có liên thông.
		\item Có chu trình.
	\end{enumerate}
	
	Nếu thỏa cả hai điều kiện trên thì đó là một cây khung hợp lệ.
	
	\vspace{1em}
	\textbf{Bước 4: Đếm số lượng cây khung}
	
	\textbf{Tập cạnh ban đầu:}
	\[
	E = \{
	(1, 2), (1, 4), (1, 6), (1, 3),
	(2, 4), (2, 5),
	(3, 6), (4, 6), (4, 7), (6, 7)
	\}
	\]
	\textbf{Tập đỉnh:}
	\[
	V = \{1, 2, 3, 4, 5, 6, 7\}, \quad |V| = 7
	\]
	
	\begin{algorithm}[H]
		\begin{algorithmic}[1]
			\State $count \gets 0$
			\ForAll{$S \in \binom{E}{6}$} \Comment{Duyệt qua tất cả các tổ hợp 6 cạnh}
			\State Khởi tạo đồ thị $G = (V, S)$
			\If{$G$ là liên thông và $|S| = |V| - 1$}
			\State $count \gets count + 1$
			\EndIf
			\EndFor
			\State \textbf{return} $count$
		\end{algorithmic}
	\end{algorithm}
	
	Sau khi kiểm tra tất cả 210 tổ hợp, ta đếm được tổng số cây khung là:
	
	\[
	\boxed{64}
	\]
	
	\vspace{1em}
	$\rightarrow$ Số lượng cây khung của đồ thị vô hướng thu được từ Hình 1.30 là:
	
	\[
	\boxed{64}
	\]
	
	\subsubsection*{1.4 Extend the adjacency matrix graph representation by replacing those operations having an edge as argument or giving an edge or a list of edges as result, by corresponding operations having as argument or giving as result the source and target vertices of the edge or edges: $G.del_edge(v,w)$, $G.edges()$, $G.incoming(v)$, $G.outgoing(v)$, $G.source(v,w)$, and $G.target(v,w)$.}
	
	\begin{itemize}
		\item \textbf{\texttt{G.del\_edge(v, w)}}: Xóa cạnh từ đỉnh $v$ đến đỉnh $w$.
		
		\begin{itemize}
			\item Với ma trận kề $A$, thực hiện: 
			\[
			A[v][w] \gets 0
			\]
		\end{itemize}
		
		\item \textbf{\texttt{G.edges()}}: Trả về danh sách các cặp đỉnh $(v, w)$ sao cho tồn tại cạnh từ $v$ đến $w$.
		
		\begin{itemize}
			\item Duyệt toàn bộ ma trận kề $A$, với mỗi $v,w$ thỏa:
			\[
			A[v][w] \ne 0 \Rightarrow \text{thêm } (v, w) \text{ vào danh sách}
			\]
		\end{itemize}
		
		\item \textbf{\texttt{G.incoming(v)}}: Trả về danh sách các đỉnh $u$ sao cho tồn tại cạnh từ $u$ đến $v$.
		
		\begin{itemize}
			\item Duyệt cột $v$ trong ma trận kề $A$:
			\[
			A[u][v] \ne 0 \Rightarrow u \in \text{incoming}(v)
			\]
		\end{itemize}
		
		\item \textbf{\texttt{G.outgoing(v)}}: Trả về danh sách các đỉnh $w$ sao cho tồn tại cạnh từ $v$ đến $w$.
		
		\begin{itemize}
			\item Duyệt hàng $v$ trong ma trận kề $A$:
			\[
			A[v][w] \ne 0 \Rightarrow w \in \text{outgoing}(v)
			\]
		\end{itemize}
		
		\item \textbf{\texttt{G.source(v, w)}}: Trả về đỉnh đầu của cạnh nối từ $v$ đến $w$ (nếu cạnh tồn tại).
		
		\begin{itemize}
			\item Nếu $A[v][w] \ne 0$, thì:
			\[
			\text{return } v
			\]
		\end{itemize}
		
		\item \textbf{\texttt{G.target(v, w)}}: Trả về đỉnh cuối của cạnh nối từ $v$ đến $w$ (nếu cạnh tồn tại).
		
		\begin{itemize}
			\item Nếu $A[v][w] \ne 0$, thì:
			\[
			\text{return } w
			\]
		\end{itemize}
		
	\end{itemize}
	
	\subsubsection*{1.5 Extend the first-child, next-sibling tree representation, in order to support the collection of basic operations but $T.root()$, $T.number_of_children(v)$,and $T.children(v)$ in $O(1)$ time.}
	
	Biểu diễn cây theo mô hình \textbf{con đầu tiên - anh em kế tiếp (first-child, next-sibling)} là một cách hiệu quả để biểu diễn cây đa phân nhánh (n-ary tree) bằng cấu trúc dữ liệu nhị phân. Trong mô hình này, mỗi nút chỉ giữ hai con trỏ: một trỏ đến người con đầu tiên của nó, và một trỏ đến người anh em kế tiếp trong danh sách các con.
	
	Tuy nhiên, với biểu diễn cơ bản này, một số thao tác thường gặp lại không thể thực hiện trong thời gian hằng số $O(1)$, chẳng hạn như:
	
	\begin{itemize}
		\item \texttt{T.root()}: lấy nút gốc của cây.
		\item \texttt{T.number\_of\_children(v)}: đếm số con trực tiếp của một nút $v$.
		\item \texttt{T.children(v)}: truy xuất danh sách các con của nút $v$.
	\end{itemize}
	
	Nguyên nhân là bởi vì để lấy danh sách con hoặc đếm số lượng con của một nút, ta cần duyệt qua chuỗi các nút thông qua con trỏ \texttt{next-sibling}, nên độ phức tạp thời gian là $O(k)$ với $k$ là số con.
	
	\vspace{0.5em}
	
	Để thực hiện các thao tác trên trong thời gian $O(1)$, ta có thể mở rộng cấu trúc của mỗi nút trong cây bằng cách bổ sung thêm thông tin phụ trợ:
	
	\begin{itemize}
		\item \textbf{\texttt{num\_children}}: một biến nguyên lưu số lượng con trực tiếp của nút.
		\item \textbf{\texttt{children\_list}} (tuỳ chọn): một danh sách (hoặc mảng) chứa con trỏ đến tất cả các con của nút, được cập nhật mỗi khi thêm hoặc xoá nút con.
		\item \textbf{\texttt{parent}} (nếu cần): con trỏ đến nút cha, giúp thuận tiện cho việc truy xuất gốc hoặc duyệt ngược lên trên cây.
	\end{itemize}
	
	Khi đó:
	
	\begin{itemize}
		\item \texttt{T.root()}: có thể lưu con trỏ đến nút gốc ngay trong cấu trúc cây $\Rightarrow O(1)$.
		\item \texttt{T.number\_of\_children(v)}: trả về giá trị \texttt{v.num\_children} $\Rightarrow O(1)$.
		\item \texttt{T.children(v)}: truy xuất trực tiếp \texttt{v.children\_list} nếu tồn tại $\Rightarrow O(1)$.
	\end{itemize}
	
	\vspace{0.5em}
	Như vậy, với một chút mở rộng hợp lý trong cấu trúc của mỗi nút, ta hoàn toàn có thể giữ nguyên lợi ích biểu diễn đơn giản của mô hình \textit{first-child, next-sibling} mà vẫn đảm bảo hiệu năng cao cho các thao tác truy xuất thông dụng, đạt độ phức tạp thời gian $O(1)$.
	
	\subsubsection*{1.6 Show how to double check that the graph-based representation of a tree is indeed a tree, in time linear in the size of the tree.}
	
	Khi một cây được biểu diễn dưới dạng đồ thị có hướng (graph-based representation), đặc biệt là bằng danh sách cạnh hoặc danh sách kề, chúng ta cần một cách để kiểm tra xem cấu trúc đó có thực sự là một cây hay không. Ta cần thực hiện phép kiểm tra này trong thời gian tuyến tính theo kích thước của cây (số đỉnh và số cạnh).
	
	\vspace{1em}
	Nếu một đồ thị có hướng là một cây nếu thỏa mãn đồng thời các điều kiện sau:
	
	\begin{enumerate}
		\item Đồ thị có đúng $n$ đỉnh và $n-1$ cạnh.
		\item Có đúng một nút gốc (root): chỉ có một đỉnh có bậc vào bằng $0$.
		\item Mỗi đỉnh (trừ gốc) có đúng một nút cha: tức là mỗi đỉnh có bậc vào đúng bằng $1$.
		\item Đồ thị không chứa chu trình và liên thông (mọi đỉnh đều được nối với gốc qua một chuỗi cạnh).
	\end{enumerate}
	
	\vspace{1em}
	
	Giả sử đồ thị được biểu diễn bằng danh sách kề với $n$ đỉnh và $m$ cạnh.
	
	\begin{enumerate}
		\item Kiểm tra số cạnh: Nếu $m \ne n - 1$, thì không thể là cây.
		
		\item Tính bậc vào (in-degree) của mỗi đỉnh:
		\begin{itemize}
			\item Duyệt qua tất cả các cạnh $(u, v)$, với mỗi cạnh ta tăng $in\_degree[v]$ thêm $1$.
			\item Sau đó:
			\begin{itemize}
				\item Đếm số đỉnh có $in\_degree = 0$ (gốc): phải đúng 1 đỉnh.
				\item Kiểm tra các đỉnh còn lại có $in\_degree = 1$.
				\item Nếu không đúng, thì không phải cây.
			\end{itemize}
		\end{itemize}
		
		\item Kiểm tra liên thông và không có chu trình:
		\begin{itemize}
			\item Bắt đầu từ đỉnh gốc, thực hiện duyệt đồ thị (DFS hoặc BFS).
			\item Trong quá trình duyệt:
			\begin{itemize}
				\item Nếu gặp lại một đỉnh đã thăm $\Rightarrow$ có chu trình $\Rightarrow$ không phải cây.
				\item Sau khi duyệt xong, kiểm tra số lượng đỉnh đã thăm phải đúng bằng $n$ (tức là liên thông).
			\end{itemize}
		\end{itemize}
	\end{enumerate}
	
	\vspace{1em}
	\textbf{Độ phức tạp thời gian:} Mọi bước trong giải thuật đều chạy trong thời gian $O(n + m)$, mà với cây thì $m = n - 1$, do đó tổng thể là $O(n)$.
	
	\vspace{1em}
	Như vậy, ta có thể kiểm tra một cách chắc chắn rằng một đồ thị có hướng có biểu diễn đúng là một cây hay không bằng các bước kiểm tra đơn giản về số cạnh, bậc vào, chu trình và liên thông, tất cả đều thực hiện được trong thời gian tuyến tính.
	
	\subsubsection*{ Exercise 1.3 Implement algorithms to generate the path graph $P_n$, the circle graph $C_n$, and the wheel graph Wn on n vertices, using the collection of 32 abstract operations from Sect.}
	
	\begin{itemize}
		\item \textbf{Đồ thị đường đi $P_n$} là một chuỗi các đỉnh nối liên tiếp, tức là có $n$ đỉnh và $n - 1$ cạnh, với mỗi cặp đỉnh $v_i$ và $v_{i+1}$ được nối bằng một cạnh.
		
		\item \textbf{Đồ thị chu trình $C_n$} là một đồ thị đường đi $P_n$ được “đóng vòng”, nghĩa là thêm cạnh nối giữa đỉnh đầu tiên và đỉnh cuối cùng.
		
		\item \textbf{Đồ thị bánh xe $W_n$} được xây dựng từ đồ thị chu trình $C_{n-1}$ bằng cách thêm một đỉnh trung tâm (gọi là trục bánh xe) và nối nó với tất cả các đỉnh trên chu trình.
	\end{itemize}
	
	Giả sử chúng ta có sẵn các phép toán trừu tượng như:
	\begin{itemize}
		\item \texttt{G.add\_vertex()}: thêm đỉnh mới và trả về định danh của đỉnh.
		\item \texttt{G.add\_edge(u, v)}: thêm cạnh nối hai đỉnh $u$ và $v$.
	\end{itemize}
	
	\textbf{Với thuật toán sinh $P_n$:}
	\begin{enumerate}
		\item Khởi tạo đồ thị rỗng $G$.
		\item Thêm $n$ đỉnh: $v_0, v_1, \ldots, v_{n-1}$.
		\item Với mỗi $i$ từ $0$ đến $n-2$, thêm cạnh từ $v_i$ đến $v_{i+1}$.
	\end{enumerate}
	
	\textbf{Với thuật toán sinh $C_n$:}
	\begin{itemize}
		\item Thực hiện giống như $P_n$.
		\item Thêm cạnh từ $v_{n-1}$ đến $v_0$ để tạo chu trình.
	\end{itemize}
	
	\textbf{Với thuật toán sinh $W_n$:}
	\begin{enumerate}
		\item Tạo chu trình $C_{n-1}$ gồm các đỉnh $v_0, \ldots, v_{n-2}$.
		\item Thêm đỉnh trung tâm $c$.
		\item Với mỗi đỉnh $v_i$ trên chu trình, thêm cạnh giữa $c$ và $v_i$.
	\end{enumerate}
	
	Như vậy, bằng cách sử dụng các thao tác trừu tượng như thêm đỉnh và thêm cạnh, ta có thể xây dựng ba loại đồ thị $P_n$, $C_n$, và $W_n$ một cách tuần tự, đơn giản và rõ ràng, với chi phí thời gian tuyến tính theo số đỉnh.
	
	\subsubsection*{Exercise 1.4 Implement an algorithm to generate the complete graph $K_n$ on $n$ vertices and the complete bipartite graph $K_{p,q}$ with $p$ + $q$ vertices, using the collection of 32 abstract operations from Sect.}
	
	\begin{itemize}
		\item Với đồ thị đầy đủ \(K_n\), ta:
		\begin{enumerate}
			\item Tạo \(n\) đỉnh bằng các lệnh \texttt{G.add\_vertex()}.
			\item Duyệt qua tất cả các cặp \((i, j)\) với \(i \ne j\), rồi thêm cạnh nối giữa chúng bằng \texttt{G.add\_edge(i, j)}.
		\end{enumerate}
		
		\item Với đồ thị hai phía đầy đủ \(K_{p,q}\), ta:
		\begin{enumerate}
			\item Tạo \(p+q\) đỉnh: \(p\) đỉnh cho tập trái \(U\), và \(q\) đỉnh cho tập phải \(V\).
			\item Duyệt tất cả các cặp \((u, v)\) với \(u \in U, v \in V\), rồi thêm cạnh nối giữa \(u\) và \(v\).
		\end{enumerate}
	\end{itemize}
	
	Độ phức tạp thời gian:
	
	\begin{itemize}
		\item Với \(K_n\): \(\mathcal{O}(n^2)\) do cần duyệt tất cả cặp đỉnh.
		\item Với \(K_{p,q}\): \(\mathcal{O}(pq)\) vì có đúng \(pq\) cạnh cần thêm.
	\end{itemize}

	\subsubsection*{Exercise 1.5 Implement the extended adjacency matrix graph representation given in Problem 1.4, wrapped in a Python class, using Python lists together with the internal numbering of the vertices}
	
	Ý tưởng:
	
	Ta xây dựng một lớp \texttt{Graph} với các đặc điểm sau:
	
	\begin{itemize}
		\item Sử dụng ma trận kề dạng danh sách 2 chiều (list of lists) để lưu các cạnh. 
		\item Với đồ thị có $n$ đỉnh, ma trận kề là ma trận $n \times n$, phần tử tại vị trí $(i, j)$ có giá trị \texttt{True} nếu có cạnh từ đỉnh $i$ đến đỉnh $j$.
		\item Dựa vào ý tưởng đó, ta có thể có các thao tác sau:
		\begin{enumerate}
			\item \texttt{add\_edge(v, w)}: thêm cạnh từ $v$ đến $w$
			\item \texttt{del\_edge(v, w)}: xoá cạnh từ $v$ đến $w$
			\item \texttt{edges()}: trả về danh sách các cạnh dưới dạng cặp $(v, w)$
			\item \texttt{incoming(v)}: trả về danh sách các đỉnh $u$ sao cho có cạnh từ $u \to v$
			\item \texttt{outgoing(v)}: trả về danh sách các đỉnh $w$ sao cho có cạnh từ $v \to w$
			\item \texttt{source(v, w)}: trả về đỉnh nguồn của cạnh $(v, w)$ nếu tồn tại
			\item \texttt{target(v, w)}: trả về đỉnh đích của cạnh $(v, w)$ nếu tồn tại
		\end{enumerate}
	\end{itemize}
	
	\subsubsection*{Exercise 1.6 Enumerate all perfect matchings in the complete bipartite graph $K_{p,q}$ on $p$ + $q$ vertices.}
	
	\begin{itemize}
		\item Đồ thị $K_{p,q}$ gồm hai tập đỉnh $U$ và $V$ với:
		\begin{itemize}
			\item $|U| = p$, $|V| = q$
			\item Mỗi đỉnh trong $U$ nối với mọi đỉnh trong $V$
		\end{itemize}
		\item Ghép cặp hoàn hảo (perfect matching) là tập các cạnh sao cho:
		\begin{itemize}
			\item Mỗi đỉnh chỉ thuộc đúng một cạnh
			\item Tức là toàn bộ các đỉnh được ghép đôi hoàn toàn giữa hai tập
		\end{itemize}
		\item Điều kiện cần: \textbf{phải có $p = q$}, nếu không sẽ không thể có ghép cặp hoàn hảo.
	\end{itemize}
	
	Ta tìm số lượng ghép cặp hoàn hảo như sau:
	
	Với $p = q = n$, số ghép cặp hoàn hảo trong $K_{n,n}$ chính là số hoán vị của $n$ phần tử:
	\[
	\text{Số lượng} = n!
	\]
	Mỗi hoán vị tương ứng với một cách ghép từng đỉnh $u_i \in U$ với một đỉnh $v_{\sigma(i)} \in V$.
	
	\subsubsection*{Exercise 1.7 Implement an algorithm to generate the complete binary tree with $n$ nodes, using the collection of 13 abstract operations from Sect.}
	
	Bài toán yêu cầu xây dựng cây nhị phân đầy đủ với đúng $n$ nút, sử dụng bộ 13 phép toán trừu tượng được đề cập trong Mục 1.3. Một cây nhị phân đầy đủ (complete binary tree) là cây trong đó các mức, ngoại trừ mức cuối cùng, đều được điền đầy đủ; các nút ở mức cuối nằm càng bên trái càng tốt.
	
	\begin{itemize}
		\item Bắt đầu bằng cách tạo nút gốc (root).
		\item Sử dụng một hàng đợi để duyệt các nút theo thứ tự mức (level-order traversal).
		\item Với mỗi nút trong hàng đợi, lần lượt chèn nút con trái và nút con phải (nếu tổng số nút chưa đạt $n$).
		\item Dừng lại khi đã tạo đủ $n$ nút.
	\end{itemize}
	
	Ta có thể mô phỏng trừu tượng như sau:
	
	\begin{enumerate}
		\item Gọi \texttt{T.Create()} để tạo cây rỗng.
		\item Gọi \texttt{r = T.AddRoot()} để thêm nút gốc, tăng số lượng nút hiện tại lên 1.
		\item Khởi tạo hàng đợi \texttt{Q} và thêm nút gốc \texttt{r} vào hàng đợi.
		\item Trong khi số nút hiện tại nhỏ hơn $n$, thực hiện:
		\begin{itemize}
			\item Lấy nút \texttt{v} từ hàng đợi (\texttt{v = Q.pop()}).
			\item Nếu tổng số nút chưa đủ $n$, chèn nút trái: \texttt{l = T.AddLeft(v)}, thêm \texttt{l} vào hàng đợi.
			\item Nếu tổng số nút chưa đủ $n$, chèn nút phải: \texttt{r = T.AddRight(v)}, thêm \texttt{r} vào hàng đợi.
		\end{itemize}
	\end{enumerate}
	
	\subsubsection*{Exercise 1.8 Implement an algorithm to generate random trees with $n$ nodes, using the collection of 13 abstract operations from Sect. 1.3. Give the time and space complexity of the algorithm}
	
	Ta sẽ sinh tuần tự $n$ nút bằng các phép toán trừu tượng, và liên kết chúng lại thành một cây bằng cách chọn ngẫu nhiên cha cho từng nút mới sinh. Thuật toán thực hiện như sau:
	
	\begin{itemize}
		\item Bước 1: Khởi tạo nút gốc $r \leftarrow \texttt{makeNode()}$.
		\item Bước 2: Khởi tạo danh sách các nút đã có: $V \leftarrow [r]$.
		\item Bước 3: Với mỗi $i$ từ $2$ đến $n$:
		\begin{itemize}
			\item Sinh nút mới: $u \leftarrow \texttt{makeNode()}$
			\item Chọn ngẫu nhiên một nút $v \in V$
			\item Gắn $u$ làm con của $v$: $\texttt{addChild}(v, u)$
			\item Thêm $u$ vào $V$
		\end{itemize}
	\end{itemize}
	
	\subsubsection*{Exercise 1.9 Give an implementation of operation $T.previous\_sibling(v)$ using the array-of-parents tree representation.}
	
	Ta cần thao tác previous\_sibling(v) để tìm đỉnh anh/chị em nằm bên trái (trước) của đỉnh $v$ trên cây.
	
	\begin{itemize}
		\item Gọi $p = P[v]$ là cha của đỉnh $v$.
		\item Duyệt qua tất cả các đỉnh $u$ từ $1$ đến $v-1$:
		\begin{itemize}
			\item Nếu $P[u] = p$ thì $u$ là anh/chị em của $v$ và đứng trước $v$.
			\item Ghi nhận giá trị $u$ lớn nhất thỏa điều kiện này.
		\end{itemize}
		\item Kết quả là $u$ lớn nhất nhỏ hơn $v$ có cùng cha với $v$.
	\end{itemize}
	
	\subsubsection*{ Exercise 1.10 Implement the extended first-child,next-sibling tree representation of Problem 1.5, wrapped in a Python class, using Python lists together with the internal numbering of the nodes}
	
	Để đạt được thời gian O(1) cho các thao tác trên, ta mở rộng cấu trúc dữ liệu như sau:
	
	\begin{itemize}
		\item Với mỗi nút \(v\), lưu:
		\begin{itemize}
			\item \texttt{first\_child[v]}: con đầu tiên của \(v\)
			\item \texttt{next\_sibling[v]}: anh em kế tiếp của \(v\)
			\item \texttt{parent[v]}: cha của \(v\)
			\item \texttt{num\_children[v]}: số con của \(v\) (cập nhật khi thêm con)
			\item \texttt{children\_list[v]}: danh sách con của \(v\) để truy cập O(1)
		\end{itemize}
		\item Duy trì \texttt{root}: đỉnh gốc
	\end{itemize}
	
	\subsection*{Bài toán 6: Viết chương trình {\sf C{\tt/}C++, Python} để giải bài toán tree edit distance problem bằng cách sử dụng: (a) Backtracking. (b) Branch-\&-bound. (c) Divide-\&-conquer -- chia để trị. (d) Dynamic programming -- Quy hoạch động.}
	
	Giả sử có hai cây có gốc $T_1$ và $T_2$, ta cần xác định chi phí tối thiểu để biến $T_1$ thành $T_2$ qua một chuỗi các phép chỉnh sửa.
	
	\subsubsection*{(a) Backtracking -- quay lui}
	
	\begin{itemize}
		\item Duyệt tất cả các cách kết hợp các phép toán giữa các nút trong hai cây.
		\item Với mỗi cặp nút $(u,v)$ từ $T_1$ và $T_2$, thử:
		\begin{itemize}
			\item Ghép $u$ với $v$ (nếu nhãn giống nhau thì chi phí là 0, nếu khác thì là 1).
			\item Bỏ $u$ (chi phí là 1).
			\item Thêm $v$ (chi phí là 1).
		\end{itemize}
		\item Đệ quy xử lý các cây con tương ứng.
		\item Chọn phương án có tổng chi phí nhỏ nhất.
	\end{itemize}
	
	\subsubsection*{(b) Branch-and-bound -- nhánh cận}
	\begin{itemize}
		\item Tương tự như Backtracking nhưng có thêm phần cắt nhánh.
		\item Tại mỗi bước, tính \textit{ước lượng tốt nhất} còn lại (lower bound) của chi phí.
		\item Nếu tổng chi phí hiện tại + ước lượng còn lại > chi phí tốt nhất tìm được → \textbf{cắt nhánh}.
	\end{itemize}
	
	\subsubsection*{(c) Divide-and-conquer -- chia để trị}
	
	\begin{itemize}
		\item Phân chia cây thành các cây con (theo từng con của gốc).
		\item Đệ quy giải bài toán TED cho các cây con tương ứng.
		\item Kết hợp kết quả các cây con để tính TED tổng thể.
	\end{itemize}
	
	\subsubsection*{(d) Dynamic programming -- Quy hoạch động}
	
	\begin{itemize}
		\item Xây dựng bảng lưu TED giữa các cặp cây con của $T_1$ và $T_2$.
		\item Duyệt cây theo thứ tự hậu tố (post-order) để xác định các cây con.
		\item Với mỗi cặp cây con $(T_1[i], T_2[j])$, tính TED dựa trên các phép:
		\begin{itemize}
			\item Chèn nút vào cây.
			\item Xoá nút khỏi cây.
			\item Đổi nhãn nếu không trùng.
		\end{itemize}
		\item Sử dụng bảng TED đã tính để tránh lặp lại phép tính.
	\end{itemize}
	
	\subsection*{Bài toán 7: Viết chương trình {\sf C{\tt/}C++, Python} để duyệt cây: (a) preorder traversal. (b) postorder traversal. (c) top-down traveral. (d) bottom-up traversal.}
	
	\subsubsection*{(a) Duyệt cây theo thứ tự trước (Preorder)}
	
	\begin{itemize}[itemsep=0.5em]
		\item Bắt đầu từ nút gốc.
		\item In ra nhãn của nút hiện tại.
		\item Gọi đệ quy để duyệt từng con của nút hiện tại theo thứ tự từ trái sang phải.
	\end{itemize}
	
	
	\subsubsection*{(b) Duyệt cây theo thứ tự sau (Postorder)}
	
	\begin{itemize}[itemsep=0.5em]
		\item Bắt đầu từ nút gốc.
		\item Gọi đệ quy để duyệt từng con của nút hiện tại trước.
		\item Sau khi đã duyệt hết các con, in ra nhãn của nút hiện tại.
	\end{itemize}
	
	\subsubsection*{(c) Duyệt từ trên xuống (Top-down traversal)}
	
	\begin{itemize}[itemsep=0.5em]
		\item Đây là dạng duyệt cây theo từng mức, từ gốc xuống lá.
		\item Sử dụng hàng đợi (queue) để duyệt theo chiều rộng (BFS).
		\item Thực hiện duyệt mức theo mức.
	\end{itemize}
	
	\subsubsection*{(d) Duyệt từ dưới lên (Bottom-up traversal)}
	
	\begin{itemize}[itemsep=0.5em]
		\item Duyệt cây theo chiều ngược từ lá lên gốc.
		\item Sử dụng kỹ thuật postorder traversal để đảm bảo mọi con đã được duyệt.
		\item Trong một số trường hợp, cần lưu lại độ sâu để xử lý theo từng mức từ dưới lên.
	\end{itemize}
	
	\subsection*{Bài toán 8: Let $G = (V,E)$ be a finite simple graph. Implement the breadth-first search on $G$.}
	
	Thuật toán duyệt theo chiều rộng (BFS) bắt đầu từ một đỉnh nguồn $s \in V$, sau đó:
	
	\begin{enumerate}
		\item Khởi tạo hàng đợi rỗng và đánh dấu đỉnh $s$ là đã thăm.
		\item Đưa $s$ vào hàng đợi.
		\item Trong khi hàng đợi không rỗng:
		\begin{itemize}
			\item Lấy đỉnh $u$ ra khỏi hàng đợi.
			\item Với mỗi đỉnh kề $v$ của $u$ chưa được thăm:
			\begin{itemize}
				\item Đánh dấu $v$ là đã thăm.
				\item Thêm $v$ vào hàng đợi.
			\end{itemize}
		\end{itemize}
	\end{enumerate}
	
	Thuật toán đảm bảo rằng các đỉnh được thăm theo thứ tự tăng dần khoảng cách (số cạnh) từ đỉnh nguồn $s$.
	
	\subsection*{Bài toán 9: Let $G = (V,E)$ be a finite multigraph. Implement the breadth-first search on $G$.}
	
	\begin{itemize}
		\item $V$ là tập hữu hạn các đỉnh.
		\item $E$ là tập hữu hạn các cạnh, cho phép tồn tại nhiều cạnh nối cùng một cặp đỉnh (gọi là \textit{đa cạnh}).
	\end{itemize}
	
	Như vậy, ý tưởng về BFS cho bài toán này là:
	
		\begin{itemize}
			\item Bắt đầu từ đỉnh $s$, đánh dấu nó là đã thăm.
			\item Duyệt các đỉnh kề của $s$ theo thứ tự hàng đợi (FIFO).
			\item Tiếp tục mở rộng đến các đỉnh kế tiếp trong mức (layer) kế tiếp.
		\end{itemize}
	Với đồ thị đa, nếu tồn tại nhiều cạnh nối hai đỉnh, ta chỉ tính một lần duyệt cho mỗi đỉnh (không xét lặp lại qua cùng một đỉnh).
	
	\subsection*{Bài toán 10: Let $G = (V,E)$ be a general graph. Implement the breadth-first search on $G$.}
	
	Với một đồ thị tổng quát (có thể là vô hướng hoặc có hướng, có hoặc không có chu trình):
	
	\begin{itemize}
		\item Đồ thị có thể không liên thông, do đó cần áp dụng BFS nhiều lần để đảm bảo duyệt hết toàn bộ các thành phần liên thông.
		\item Có thể sử dụng một tập hoặc mảng \texttt{visited} để đảm bảo không duyệt lại đỉnh đã thăm.
		\item Nếu đồ thị là có hướng, chỉ xét cạnh theo hướng được chỉ định.
		\item Nếu đồ thị là vô hướng, cần tránh duyệt 2 chiều cùng một cạnh.
	\end{itemize}
	
	Do đó, BFS hoạt động theo các bước sau:
	
	\begin{enumerate}
		\item Khởi tạo: Chọn một đỉnh bắt đầu $s \in V$. Đánh dấu $s$ là đã thăm và đưa vào hàng đợi.
		
		\item Lặp: Trong khi hàng đợi chưa rỗng:
		\begin{itemize}
			\item Lấy một đỉnh $u$ ra khỏi hàng đợi.
			\item Với mỗi đỉnh kề $v$ của $u$:
			\begin{itemize}
				\item Nếu $v$ chưa được thăm, đánh dấu $v$ và đưa vào hàng đợi.
			\end{itemize}
		\end{itemize}
		
		\item Tiếp tục cho tới khi hàng đợi rỗng.
	\end{enumerate}
	
	\subsection*{Bài toán 11:  Let $G = (V,E)$ be a finite simple graph. Implement the depth-first search on $G$.}
	
	Cho $G = (V, E)$ là một đồ thị đơn hữu hạn, trong đó:
	\begin{itemize}
		\item $V$ là tập các đỉnh.
		\item $E$ là tập các cạnh, không có cạnh khuyên (self-loop) và không có cạnh lặp (multiple edges).
	\end{itemize}
	
	Khi đó, thuật toán DFS sẽ duyệt như sau:
	
	\begin{itemize}
		\item Từ một đỉnh ban đầu $s$, thuật toán đi theo một nhánh bất kỳ cho đến khi không còn đi tiếp được (đạt tới đỉnh đã được thăm hoặc không còn đỉnh kề).
		\item Sau đó, quay lại đỉnh trước đó (backtracking) và tiếp tục thử các nhánh còn lại.
		\item Quá trình lặp lại cho đến khi mọi đỉnh liên thông với $s$ được thăm.
	\end{itemize}
	
	\subsection*{Bài toán 12: Let $G = (V,E)$ be a finite multigraph. Implement the depth-first search on $G$.}
	
		Với một đồ thị đa $G = (V, E)$ được biểu diễn dưới dạng danh sách kề, hoặc danh sách cạnh (mỗi cạnh có thể xuất hiện nhiều lần nếu là đa cạnh).
		
		Như vậy ý tưởng sẽ là:
		\begin{enumerate}
			\item Khởi tạo một tập rỗng \texttt{visited} để đánh dấu các đỉnh đã được thăm.
			\item Chọn một đỉnh bắt đầu $v \in V$, tiến hành:
			\begin{itemize}
				\item Đánh dấu $v$ là đã thăm.
				\item Với mỗi cạnh $(v, u) \in E$ (kể cả khi có nhiều cạnh trùng nhau), nếu $u$ chưa được thăm thì đệ quy gọi DFS($u$).
			\end{itemize}
			\item Quá trình kết thúc khi tất cả các đỉnh có thể đi đến từ $v$ đã được thăm.
		\end{enumerate}
		
	\subsection*{Bài toán 13: Let $G = (V,E)$ be a general graph. Implement the depth-first search on $G$.}
	
	Duyệt tất cả các đỉnh của đồ thị xuất phát từ một đỉnh cho trước, bằng cách thăm các đỉnh kề chưa được thăm trước khi quay lui.
	
	\begin{itemize}
		\item Khởi tạo một mảng hoặc tập \texttt{visited} để đánh dấu các đỉnh đã được duyệt.
		\item Chọn một đỉnh bắt đầu $u \in V$.
		\item Đánh dấu $u$ là đã thăm.
		\item Duyệt qua tất cả các đỉnh kề với $u$ (bao gồm cả đa cạnh hoặc cạnh khuyên nếu có).
		\begin{itemize}
			\item Nếu một đỉnh $v$ kề với $u$ chưa được thăm, gọi đệ quy DFS trên $v$.
		\end{itemize}
		\item Quá trình tiếp tục cho đến khi tất cả các đỉnh liên thông với đỉnh bắt đầu đã được thăm.
		\item Nếu cần duyệt toàn bộ đồ thị (kể cả thành phần rời rạc), lặp lại DFS cho các đỉnh chưa thăm còn lại.
	\end{itemize}
	
	\section*{Project 5: Shortest Path Problems on Graphs -- Đồ Án 5: Các Bài Toán Tìm Đường Đi Ngắn Nhất Trên Đồ Thị}
	
	\subsection*{Bài toán 14: Let $G = (V,E)$ be a finite simple graph. Implement the Dijkstra's algorithm to find the shortest path problem on $G$.}
	
	Ta có ý tưởng như sau:
	
	\begin{enumerate}
		\item Khởi tạo một mảng khoảng cách $dist[v]$ với mọi $v \in V$, trong đó:
		\[
		dist[s] = 0, \quad dist[v] = \infty \text{ với mọi } v \neq s.
		\]
		\item Khởi tạo một tập đỉnh $Q$ chứa tất cả các đỉnh trong đồ thị (chưa được xét).
		\item Lặp cho đến khi $Q$ rỗng:
		\begin{enumerate}
			\item Chọn đỉnh $u \in Q$ có $dist[u]$ nhỏ nhất (đỉnh gần nhất chưa xét).
			\item Loại bỏ $u$ khỏi $Q$.
			\item Với mỗi đỉnh kề $v$ của $u$, nếu $v \in Q$ và tồn tại cạnh $(u, v)$ có trọng số $w(u, v)$, thì kiểm tra:
			\[
			\text{Nếu } dist[v] > dist[u] + w(u, v) \text{ thì cập nhật } dist[v] = dist[u] + w(u, v).
			\]
		\end{enumerate}
		\item Sau khi thuật toán kết thúc, $dist[v]$ là độ dài đường đi ngắn nhất từ $s$ đến $v$.
	\end{enumerate}
	
	\subsection*{Bài toán 15: Let $G = (V,E)$ be a finite multigraph. Implement the Dijkstra's algorithm to find the shortest path problem on $G$.}
	
	Ta có ý tưởng như sau:
	
	\begin{enumerate}
		\item Gán giá trị khoảng cách từ đỉnh nguồn $s$ đến tất cả các đỉnh là vô cùng, trừ $s$ có khoảng cách là $0$.
		\item Sử dụng một hàng đợi ưu tiên (priority queue) để liên tục chọn đỉnh $u$ có khoảng cách nhỏ nhất hiện tại.
		\item Với mỗi đỉnh $u$ được chọn, duyệt tất cả các \textbf{cạnh} kề $(u,v,w)$ (bao gồm các cạnh song song nếu có) và cập nhật nếu tìm thấy một đường đi tốt hơn:
		\[
		\text{Nếu } \text{dist}[v] > \text{dist}[u] + w \text{ thì cập nhật: } \text{dist}[v] = \text{dist}[u] + w
		\]
		\item Lặp lại bước trên cho đến khi hàng đợi rỗng.
	\end{enumerate}
	
	\subsection*{Bài toán 16: Let $G = (V,E)$ be a general graph. Implement the Dijkstra's algorithm to find the shortest path problem on $G$.}
	
	Ta có ý tưởng như sau:
	
	\begin{itemize}
		\item Gán giá trị khoảng cách ban đầu từ đỉnh xuất phát $s$ đến tất cả các đỉnh là vô cùng ($\infty$), riêng $d(s) = 0$.
		
		\item Sử dụng một hàng đợi ưu tiên (priority queue) hoặc một tập đỉnh chưa xét để lần lượt chọn đỉnh $u$ có khoảng cách tạm thời nhỏ nhất.
		
		\item Với mỗi đỉnh $u$ được chọn:
		\begin{itemize}
			\item Duyệt qua tất cả các cạnh $(u, v) \in E$ xuất phát từ $u$ (kể cả cạnh song song, nếu có).
			\item Nếu đường đi mới từ $s$ đến $v$ thông qua $u$ ngắn hơn giá trị hiện tại của $d(v)$, thì cập nhật lại $d(v)$.
		\end{itemize}
		
		\item Lặp lại cho đến khi không còn đỉnh nào trong hàng đợi/tập đỉnh chưa xét.
	\end{itemize}
\end{document}