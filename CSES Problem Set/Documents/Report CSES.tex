\documentclass{article}
\usepackage{amsmath}
\usepackage{amssymb}
\usepackage{enumitem}
\usepackage{tikz}
\usetikzlibrary{graphs}
\usepackage[utf8]{inputenc}
\usepackage[T5]{fontenc}
\usepackage[vietnamese]{babel}

\begin{document}
	
	\section{BuildingRoads}
	
	Cho $N$ thành phố được đánh số từ $1$ đến $N$, và $M$ con đường đã xây giữa một số cặp thành phố. Mỗi con đường nối hai thành phố $a$ và $b$. 
	
	Mục tiêu là thêm vào một số tối thiểu các con đường sao cho mọi thành phố đều nằm trong cùng một thành phần liên thông.
		
	\subsection*{1. Cấu trúc DSU}
	Mỗi thành phố ban đầu là một tập riêng biệt.
	
	\begin{itemize}
		\item Mảng \texttt{ds[i]} lưu thông tin về gốc của tập chứa thành phố $i$.
		\item Hàm \texttt{find(u)} trả về gốc của tập chứa $u$, đồng thời áp dụng \textit{path compression}:
		\[
		\text{find}(u) = 
		\begin{cases}
			u & \text{nếu } ds[u] < 0 \\
			\text{find}(ds[u]) & \text{ngược lại}
		\end{cases}
		\]
		\item Hàm \texttt{merge(u, v)} hợp nhất hai tập nếu $u$ và $v$ thuộc hai tập khác nhau:
		\[
		\text{merge}(u, v) = 
		\begin{cases}
			\text{false} & \text{nếu } \text{find}(u) = \text{find}(v) \\
			\text{gộp hai tập và trả về true} & \text{ngược lại}
		\end{cases}
		\]
	\end{itemize}
	
	\subsection*{2. Thuật toán chính}
	\begin{enumerate}
		\item Khởi tạo: \texttt{ds[i] = -1} với mọi $i$ từ $1$ đến $N$.
		\item Với mỗi cạnh $(a, b)$ đầu vào, gọi \texttt{merge(a, b)} để nối các thành phố đã liên thông.
		\item Duyệt từ $i = 1$ đến $N - 1$:
		\begin{itemize}
			\item Nếu \texttt{merge(i, i+1)} thành công, nghĩa là $i$ và $i+1$ chưa liên thông, thì thêm cạnh $(i, i+1)$ vào danh sách kết quả.
		\end{itemize}
		\item In ra số đường thêm vào và danh sách các đường đó.
	\end{enumerate}
	
	\subsection*{3. Độ phức tạp}
	Với kỹ thuật \textit{path compression} và \textit{union by size}, mỗi phép \texttt{find} hay \texttt{merge} có độ phức tạp gần $\mathcal{O}(1)$ (chính xác là $\mathcal{O}(\alpha(N))$ với $\alpha$ là hàm nghịch đảo Ackermann).
	
	Tổng độ phức tạp:
	\[
	\mathcal{O}(M \cdot \alpha(N) + N \cdot \alpha(N)) \approx \mathcal{O}(N)
	\]
	
	\section{CountingRooms}
	\subsection*{Mô tả bài toán}
	Cho một mê cung dưới dạng lưới $N \times M$ với mỗi ô là:
	\begin{itemize}
		\item \texttt{\char`\.} — ô trống có thể đi vào.
		\item \texttt{\char`\#} — tường không thể đi qua.
	\end{itemize}
	
	Hai ô trống được xem là \textbf{liên thông} nếu chúng kề nhau theo hướng lên, xuống, trái, hoặc phải.
	
	\textbf{Yêu cầu:} Đếm số vùng liên thông các ô trống — gọi là \textit{số phòng}.
	\subsection*{Ý tưởng thuật toán}
	
	\begin{enumerate}
		\item Duyệt toàn bộ ma trận.
		\item Với mỗi ô chưa được thăm và là ô trống \texttt{'.'}, thực hiện thuật toán DFS để đánh dấu tất cả các ô trong vùng liên thông.
		\item Mỗi lần gọi DFS tương ứng với một phòng mới.
	\end{enumerate}
	
	\subsection*{Hàm DFS}
	Giả sử đang ở ô $(x, y)$, đánh dấu là đã thăm:
	\[
	\text{visited}[x][y] \leftarrow \text{true}
	\]
	Sau đó, thử di chuyển sang 4 hướng:
	\[
	(x + 1, y),\ (x - 1, y),\ (x, y + 1),\ (x, y - 1)
	\]
	Nếu ô mới hợp lệ, không phải tường, chưa được thăm, thì tiếp tục gọi đệ quy.
	
	\subsection*{Độ phức tạp thời gian}
	Mỗi ô được thăm nhiều nhất 1 lần, nên tổng thời gian là:
	\[
	\mathcal{O}(N \cdot M)
	\]

	\section{Labyrinth}
	\subsection*{Bài toán}
	Cho mê cung dạng lưới kích thước \( N \times M \). Mỗi ô có thể là:
	\begin{itemize}
		\item Tường (\#): không đi được
		\item Đường đi (.)
		\item Điểm bắt đầu: \( A \)
		\item Điểm kết thúc: \( B \)
	\end{itemize}
	
	\textbf{Yêu cầu:} Tìm đường đi ngắn nhất từ \( A \) đến \( B \), hoặc trả lời rằng không thể.
	
	\subsection*{Mô hình hoá bài toán}
	Mỗi ô trong lưới là một đỉnh trong đồ thị. Các đỉnh kề nhau nếu chúng là ô trống và liền kề theo 4 hướng:
	\[
	\text{Hướng dịch chuyển:}
	\begin{cases}
		\text{Xuống (D)}: (x+1, y) \\
		\text{Lên (U)}: (x-1, y) \\
		\text{Phải (R)}: (x, y+1) \\
		\text{Trái (L)}: (x, y-1)
	\end{cases}
	\]
	
	\subsection*{Giải thuật BFS}
	\begin{enumerate}
		\item Khởi tạo hàng đợi \( Q \), đánh dấu ô bắt đầu \( A \) là đã thăm.
		\item Trong khi \( Q \) chưa rỗng:
		\begin{enumerate}
			\item Lấy phần tử đầu hàng đợi: \( (x, y) \)
			\item Duyệt 4 hướng để tìm ô kề chưa thăm và không phải tường
			\item Đánh dấu đã thăm, lưu hướng đi, cập nhật khoảng cách:
			\[
			\text{dist}[x'][y'] = \text{dist}[x][y] + 1
			\]
			\item Thêm ô \( (x', y') \) vào hàng đợi
		\end{enumerate}
		\item Khi kết thúc, nếu ô \( B \) chưa được thăm, trả lời \texttt{NO}
		\item Ngược lại, in ra đường đi bằng cách lần ngược từ \( B \) về \( A \) theo mảng hướng đi \texttt{par}
	\end{enumerate}
	
	\subsection*{Độ phức tạp}
	\begin{itemize}
		\item Thời gian: \( O(N \times M) \), vì mỗi ô được duyệt tối đa 1 lần
		\item Không gian: \( O(N \times M) \) cho các mảng: \texttt{vis}, \texttt{dist}, \texttt{par}
	\end{itemize}
	
	\section{MessageRoute}
	\subsection*{Đề bài}
	Cho một đồ thị vô hướng gồm \( N \) đỉnh và \( M \) cạnh, đánh số từ 1 đến \( N \). Hãy tìm đường đi ngắn nhất từ đỉnh 1 đến đỉnh \( N \). Nếu không tồn tại, in ra \texttt{IMPOSSIBLE}.
	
	\textbf{Input:}
	\begin{itemize}
		\item Dòng đầu tiên chứa hai số nguyên \( N, M \)
		\item \( M \) dòng tiếp theo, mỗi dòng chứa hai số nguyên \( a, b \) biểu diễn cạnh nối giữa đỉnh \( a \) và đỉnh \( b \)
	\end{itemize}
	
	\textbf{Output:}
	\begin{itemize}
		\item Nếu không tồn tại đường đi từ 1 đến \( N \), in \texttt{IMPOSSIBLE}
		\item Ngược lại, in số bước và các đỉnh trên đường đi ngắn nhất
	\end{itemize}
	
	\subsection*{Ý tưởng và giải thuật}
	Sử dụng thuật toán \textbf{BFS (Tìm kiếm theo chiều rộng)} để tìm đường đi ngắn nhất từ đỉnh 1 đến đỉnh \( N \).
	
	\subsubsection{Mô hình hóa đồ thị}
	Đồ thị được biểu diễn dưới dạng danh sách kề:
	\[
	G[u] = \text{danh sách các đỉnh kề với } u
	\]
	
	\subsubsection{Biến và cấu trúc dữ liệu sử dụng}
	\begin{itemize}
		\item \texttt{vis[u]}: đánh dấu đỉnh \( u \) đã được thăm
		\item \texttt{dist[u]}: độ dài đường đi ngắn nhất từ đỉnh 1 đến \( u \)
		\item \texttt{p[u]}: đỉnh trước đó trên đường đi đến \( u \)
	\end{itemize}
	
	\subsubsection{Thuật toán BFS}
	\begin{enumerate}
		\item Khởi tạo hàng đợi \( Q \), đưa đỉnh 1 vào \( Q \), đánh dấu đã thăm
		\item Trong khi \( Q \) chưa rỗng:
		\begin{enumerate}
			\item Lấy \( u = Q.\text{front()} \), xóa khỏi hàng đợi
			\item Duyệt tất cả đỉnh \( v \) kề với \( u \)
			\item Nếu \( v \) chưa thăm:
			\begin{itemize}
				\item Cập nhật \( \text{dist}[v] = \text{dist}[u] + 1 \)
				\item Cập nhật \( p[v] = u \)
				\item Đánh dấu đã thăm và đưa vào hàng đợi
			\end{itemize}
		\end{enumerate}
		\item Sau khi BFS hoàn tất:
		\begin{itemize}
			\item Nếu \( N \) chưa được thăm \( \Rightarrow \) \texttt{IMPOSSIBLE}
			\item Ngược lại, dùng mảng \texttt{p} để truy vết đường đi từ \( N \) về 1
		\end{itemize}
	\end{enumerate}
	
	\subsection{Độ phức tạp}
	\begin{itemize}
		\item Thời gian: \( \mathcal{O}(N + M) \)
		\item Không gian: \( \mathcal{O}(N + M) \)
	\end{itemize}

	
\end{document}
